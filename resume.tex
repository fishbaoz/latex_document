%%%%%%%%%%%%%%%%%%%%%%%%%%%%%%%%%%%%%%%%%
% Medium Length Professional CV
% LaTeX Template
% Version 2.0 (8/5/13)
%
% This template has been downloaded from:
% http://www.LaTeXTemplates.com
%
% Original author:
% Trey Hunner (http://www.treyhunner.com/)
%
% Important note:
% This template requires the resume.cls file to be in the same directory as the
% .tex file. The resume.cls file provides the resume style used for structuring the
% document.
%
%%%%%%%%%%%%%%%%%%%%%%%%%%%%%%%%%%%%%%%%%

%----------------------------------------------------------------------------------------
%	PACKAGES AND OTHER DOCUMENT CONFIGURATIONS
%----------------------------------------------------------------------------------------

\documentclass{resume} % Use the custom resume.cls style

\usepackage[left=0.75in,top=0.6in,right=0.75in,bottom=0.6in]{geometry} % Document margins

\name{Zheng Bao} % Your name
\address{13-302  Xibahe Xili \\ Beijing, China} % Your address
\address{(0086)~$\cdot$~10~$\cdot$~13641090211 \\ fishbaoz@hotmail.com} % Your phone number and email

\begin{document}

%----------------------------------------------------------------------------------------
%	EDUCATION SECTION
%----------------------------------------------------------------------------------------

\begin{rSection}{Profile}

\item Software expert with more than 12 years of technical background, analyzing, developing and troubleshooting software and systems architecture
\item Extensive experience in x86 BIOS and firmware development. Strong background in computer architectures and firmware
\item Major contributor to several Open Source projects.
\item Adept at working with international corporations, interdependent departments and organizations.

\end{rSection}


%----------------------------------------------------------------------------------------
%	WORK EXPERIENCE SECTION
%----------------------------------------------------------------------------------------

\begin{rSection}{Experience}


\begin{rSubsection}{Freelance}{Nov. 2015 - present }{Firmware Engineer}{Beijing}
\item Produce more coreboot patches and work with unmerged patches in Coreboot community.
\item Porting coreboot to product designed based on COM Express specification.
\item Give coreboot training to Firmware engineers of an academy.
\item Porting LUA interpreter to STM32 Broadcom WICED Software Development Kit.
\end{rSubsection}

\begin{rSubsection}{Advanced Micro Devices, Inc (AMD)}{Sept 2004 - Nov. 2015}{Firmware Engineer, SMTS}{Beijing}
\item Lead development of Open Source firmware development
\item Bringing up and Porting coreboot to more than 20 reference design cases including blade storage server (SBB), Set-Top-Box (STB), 6-screen digital signage, NAS, low-cost PC and public reference boards covered 7 generations of AMD APU, server CPU and chipsets.
\item Contribution to Open Source projects, including Coreboot, Flashrom, SeaBIOS.
Please refer \begin{verbatim} http://review.coreboot.org/#/q/owner:zheng.bao%2540amd.com\end{verbatim}
  \subitem Porting AMD-internal code to Coreboot.
  \subitem Build AMD code as a binary module and integrate into final image.
  \subitem IP scrub the AMD code to meet the open source requirement.
  \subitem Setup the toolchain to build coreboot.
  \subitem Set up the framework of ACPI.
  \subitem Integrate AMD chipset and PSP firmware into Coreboot.
  \subitem Integrate the UEFI(Duet) payload to coreboot. Porting the AMD UEFI code to Duet.
  \subitem Porting flashrom to AMD chipset.
\item Successively promoted as Senior Engineer, MTS, SMTS for extensive contribution to both firmware development and hardware validation
\end{rSubsection}

%------------------------------------------------

\begin{rSubsection}{BLX IC Design Co., Ltd}{April 2003 - September 2004}{Software Engineer}{Beijing}
\item Maintenance and development of firmware related with Godson-I (MIPS arch),
      focusing on keyboard driver, GUI in bootloader, Chinese font displaying. 
\item Modify the MIPS toolchains to take the advantage to Godson-I.
\item Port Linux to Godson-I board. Reduce the noise of AC97. Produce the root file system.
\end{rSubsection}

%------------------------------------------------

\end{rSection}

%----------------------------------------------------------------------------------------
%	EDUCATION SECTION
%----------------------------------------------------------------------------------------

\begin{rSection}{Education}
{\bf Beijing University of Aeronautics and Astronautics } \hfill {\em June 2000 - June 2003} \\ 
M.S. in Computer Science \& Engineering \\

{\bf Posted Papers }   \\
 Multi-processor Support on Embedded Operating System
                        Journal: Computer Application and Software
 Realization of Application Development Tools of RTEMS
                        Journal: Learned Journal of BUAA


{\bf Beijing University of Aeronautics and Astronautics } \hfill {\em June 1996 - June 2000} \\ 
B.S. in Computer Science \& Engineering \\
Minor in Linguistics \smallskip \\

\end{rSection}



%----------------------------------------------------------------------------------------
%	TECHNICAL STRENGTHS SECTION
%----------------------------------------------------------------------------------------

\begin{rSection}{Technical Strengths}

\item Numerous programming languages, including C, C++, Assembler (x86/x64,
  MIPS, STM32); Scripting languages: bash, awk, sed, flex \& bison;
  Data description languages: LaTeX;

\item Development Tools: git, gerrit, Subversion, CVS,
GNU compiler suite (GCC, Binutils, gdb, gdbserver, make, Coreutils), Visual Studio, AMD SimNow, Qemu

\item Bridging between hardware and software; Extensive chipset and bring-up
experience, DDR I/II/III ram initialization, PCI, PCI-X, PCIe, SuperIO, Cardbus, USB,
SMI, EC; Specification: UEFI, ACPI.

\item Editors: Vim, Emacs

\end{rSection}

%----------------------------------------------------------------------------------------
%	EXAMPLE SECTION
%----------------------------------------------------------------------------------------

%\begin{rSection}{Section Name}

%Section content\ldots

%\end{rSection}

%----------------------------------------------------------------------------------------

\end{document}
